% Options for packages loaded elsewhere
\PassOptionsToPackage{unicode}{hyperref}
\PassOptionsToPackage{hyphens}{url}
\PassOptionsToPackage{dvipsnames,svgnames,x11names}{xcolor}
%
\documentclass[
  letterpaper,
  DIV=11,
  numbers=noendperiod]{scrartcl}

\usepackage{amsmath,amssymb}
\usepackage{lmodern}
\usepackage{iftex}
\ifPDFTeX
  \usepackage[T1]{fontenc}
  \usepackage[utf8]{inputenc}
  \usepackage{textcomp} % provide euro and other symbols
\else % if luatex or xetex
  \usepackage{unicode-math}
  \defaultfontfeatures{Scale=MatchLowercase}
  \defaultfontfeatures[\rmfamily]{Ligatures=TeX,Scale=1}
\fi
% Use upquote if available, for straight quotes in verbatim environments
\IfFileExists{upquote.sty}{\usepackage{upquote}}{}
\IfFileExists{microtype.sty}{% use microtype if available
  \usepackage[]{microtype}
  \UseMicrotypeSet[protrusion]{basicmath} % disable protrusion for tt fonts
}{}
\makeatletter
\@ifundefined{KOMAClassName}{% if non-KOMA class
  \IfFileExists{parskip.sty}{%
    \usepackage{parskip}
  }{% else
    \setlength{\parindent}{0pt}
    \setlength{\parskip}{6pt plus 2pt minus 1pt}}
}{% if KOMA class
  \KOMAoptions{parskip=half}}
\makeatother
\usepackage{xcolor}
\setlength{\emergencystretch}{3em} % prevent overfull lines
\setcounter{secnumdepth}{-\maxdimen} % remove section numbering
% Make \paragraph and \subparagraph free-standing
\ifx\paragraph\undefined\else
  \let\oldparagraph\paragraph
  \renewcommand{\paragraph}[1]{\oldparagraph{#1}\mbox{}}
\fi
\ifx\subparagraph\undefined\else
  \let\oldsubparagraph\subparagraph
  \renewcommand{\subparagraph}[1]{\oldsubparagraph{#1}\mbox{}}
\fi


\providecommand{\tightlist}{%
  \setlength{\itemsep}{0pt}\setlength{\parskip}{0pt}}\usepackage{longtable,booktabs,array}
\usepackage{calc} % for calculating minipage widths
% Correct order of tables after \paragraph or \subparagraph
\usepackage{etoolbox}
\makeatletter
\patchcmd\longtable{\par}{\if@noskipsec\mbox{}\fi\par}{}{}
\makeatother
% Allow footnotes in longtable head/foot
\IfFileExists{footnotehyper.sty}{\usepackage{footnotehyper}}{\usepackage{footnote}}
\makesavenoteenv{longtable}
\usepackage{graphicx}
\makeatletter
\def\maxwidth{\ifdim\Gin@nat@width>\linewidth\linewidth\else\Gin@nat@width\fi}
\def\maxheight{\ifdim\Gin@nat@height>\textheight\textheight\else\Gin@nat@height\fi}
\makeatother
% Scale images if necessary, so that they will not overflow the page
% margins by default, and it is still possible to overwrite the defaults
% using explicit options in \includegraphics[width, height, ...]{}
\setkeys{Gin}{width=\maxwidth,height=\maxheight,keepaspectratio}
% Set default figure placement to htbp
\makeatletter
\def\fps@figure{htbp}
\makeatother

\usepackage{booktabs}
\usepackage{caption}
\usepackage{longtable}
\KOMAoption{captions}{tableheading}
\makeatletter
\@ifpackageloaded{tcolorbox}{}{\usepackage[many]{tcolorbox}}
\@ifpackageloaded{fontawesome5}{}{\usepackage{fontawesome5}}
\definecolor{quarto-callout-color}{HTML}{909090}
\definecolor{quarto-callout-note-color}{HTML}{0758E5}
\definecolor{quarto-callout-important-color}{HTML}{CC1914}
\definecolor{quarto-callout-warning-color}{HTML}{EB9113}
\definecolor{quarto-callout-tip-color}{HTML}{00A047}
\definecolor{quarto-callout-caution-color}{HTML}{FC5300}
\definecolor{quarto-callout-color-frame}{HTML}{acacac}
\definecolor{quarto-callout-note-color-frame}{HTML}{4582ec}
\definecolor{quarto-callout-important-color-frame}{HTML}{d9534f}
\definecolor{quarto-callout-warning-color-frame}{HTML}{f0ad4e}
\definecolor{quarto-callout-tip-color-frame}{HTML}{02b875}
\definecolor{quarto-callout-caution-color-frame}{HTML}{fd7e14}
\makeatother
\makeatletter
\makeatother
\makeatletter
\makeatother
\makeatletter
\@ifpackageloaded{caption}{}{\usepackage{caption}}
\AtBeginDocument{%
\ifdefined\contentsname
  \renewcommand*\contentsname{Indice de contenidos}
\else
  \newcommand\contentsname{Indice de contenidos}
\fi
\ifdefined\listfigurename
  \renewcommand*\listfigurename{Listado de Figuras}
\else
  \newcommand\listfigurename{Listado de Figuras}
\fi
\ifdefined\listtablename
  \renewcommand*\listtablename{Listado de Tablas}
\else
  \newcommand\listtablename{Listado de Tablas}
\fi
\ifdefined\figurename
  \renewcommand*\figurename{Figura}
\else
  \newcommand\figurename{Figura}
\fi
\ifdefined\tablename
  \renewcommand*\tablename{Tabla}
\else
  \newcommand\tablename{Tabla}
\fi
}
\@ifpackageloaded{float}{}{\usepackage{float}}
\floatstyle{ruled}
\@ifundefined{c@chapter}{\newfloat{codelisting}{h}{lop}}{\newfloat{codelisting}{h}{lop}[chapter]}
\floatname{codelisting}{Listado}
\newcommand*\listoflistings{\listof{codelisting}{Listado de Listatdos}}
\makeatother
\makeatletter
\@ifpackageloaded{caption}{}{\usepackage{caption}}
\@ifpackageloaded{subcaption}{}{\usepackage{subcaption}}
\makeatother
\makeatletter
\@ifpackageloaded{tcolorbox}{}{\usepackage[many]{tcolorbox}}
\makeatother
\makeatletter
\@ifundefined{shadecolor}{\definecolor{shadecolor}{rgb}{.97, .97, .97}}
\makeatother
\makeatletter
\makeatother
\ifLuaTeX
\usepackage[bidi=basic]{babel}
\else
\usepackage[bidi=default]{babel}
\fi
\babelprovide[main,import]{spanish}
% get rid of language-specific shorthands (see #6817):
\let\LanguageShortHands\languageshorthands
\def\languageshorthands#1{}
\ifLuaTeX
  \usepackage{selnolig}  % disable illegal ligatures
\fi
\IfFileExists{bookmark.sty}{\usepackage{bookmark}}{\usepackage{hyperref}}
\IfFileExists{xurl.sty}{\usepackage{xurl}}{} % add URL line breaks if available
\urlstyle{same} % disable monospaced font for URLs
\hypersetup{
  pdftitle={Informe Resultados del Programa del Diploma IB - Convocatoria Mayo 2023},
  pdfauthor={Jorge Eduardo Baquero C.},
  pdflang={es},
  colorlinks=true,
  linkcolor={blue},
  filecolor={Maroon},
  citecolor={Blue},
  urlcolor={Blue},
  pdfcreator={LaTeX via pandoc}}

\title{Informe Resultados del Programa del Diploma IB - Convocatoria
Mayo 2023}
\usepackage{etoolbox}
\makeatletter
\providecommand{\subtitle}[1]{% add subtitle to \maketitle
  \apptocmd{\@title}{\par {\large #1 \par}}{}{}
}
\makeatother
\subtitle{Gimnasio Colombo Británico - Bilingüe Internacional}
\author{Jorge Eduardo Baquero C.}
\date{}

\begin{document}
\maketitle
\ifdefined\Shaded\renewenvironment{Shaded}{\begin{tcolorbox}[boxrule=0pt, borderline west={3pt}{0pt}{shadecolor}, enhanced, interior hidden, frame hidden, breakable, sharp corners]}{\end{tcolorbox}}\fi

\hypertarget{generalidades-y-poblaciuxf3n}{%
\subsection{1. Generalidades y
población}\label{generalidades-y-poblaciuxf3n}}

La session de mayo de 2023 fue la segunda presentación de candidatos del
GCB que cumplieron con los requisitos dispuestos para el desarrollo del
programa del Diploma IB durante dos años. Se presentaron 53 candidatos
que elaboraron en total 424 componentes de evaluación (que incluyen las
pruebas por cada asignatura, los trabajos internos de cada una de ellas,
las monografías y los componentes de evaluación de TdC).

En esta convocatoria, de los 53 candidatos, 29 fueron hombres y 24
mujeres ( Figura~\ref{fig-distribucion-aprobacion-sexo} ). \textbf{37
estudiantes de los 53 registrados (70\%) obtuvieron el Diploma IB} con
una media de puntaje para los estudiantes que obtuvieron el Diploma de
29 puntos La nota promedio de los estudiantes que obtuvieron el Diploma
fue de 4,60 por asignatura. Este resultado es inferior al obtenido en la
sesión de noviembre 2022, donde se obtuvo un 74\% de Diplomas otorgados
y una media de 4,77 por asignatura. La distribución de los estudiantes
que recibieron el Diploma fue equivalente entre ambos sexos (
Figura~\ref{fig-distribucion-aprobacion-sexo})

\begin{figure}

\begin{minipage}[t]{0.50\linewidth}

{\centering 

\raisebox{-\height}{

\includegraphics{Resultados-DP-2023_pdf_files/figure-pdf/fig-distribucion-aprobacion-sexo-1.pdf}

}

}

\subcaption{\label{fig-distribucion-aprobacion-sexo-1}Candidatos Diploma
IB 2023 por sexo}
\end{minipage}%
%
\begin{minipage}[t]{0.50\linewidth}

{\centering 

\raisebox{-\height}{

\includegraphics{Resultados-DP-2023_pdf_files/figure-pdf/fig-distribucion-aprobacion-sexo-2.pdf}

}

}

\subcaption{\label{fig-distribucion-aprobacion-sexo-2}Obtención del
Diploma por sexo}
\end{minipage}%

\caption{\label{fig-distribucion-aprobacion-sexo}Distribución y
Aprobación de candidatos por sexo}

\end{figure}

El estudiante que obtuvo el mayor puntaje fue \textbf{Juan David Uribe},
con 38 puntos totales y el estudiante con la mejor monografía (A -- en
la asignatura Español A) fue el estudiante \textbf{David Pino Rozo.} De
los 16 estudiantes que no obtuvieron el Diploma, 6 de ellos obtuvieron
puntajes superiores a 24 (puntaje mínimo para alcanzarlo) pero
desafortunadamente no cumplieron con la condición de obtener el mínimo
de puntos necesarios en sus asignaturas de nivel superior (12 puntos).
Con ellos, el porcentaje de aprobación del Diploma hubiera sido del
90\%. Los estudiantes con menores puntajes fueron Alejandro Castillo y
Daniel Triviño, quienes solo alcanzaron 20 puntos de 45.
Tabla~\ref{tbl-mejorespuntajes}

\hypertarget{tbl-mejorespuntajes}{}
\begin{longtable}[]{@{}lllll@{}}
\caption{\label{tbl-mejorespuntajes}Listado de puntajes obtenidos
Diploma IB - Mayo 2023)}\tabularnewline
\toprule()
Puesto & Estudiante & Puntos Obtenidos & Puntos Adic. & Diploma \\
\midrule()
\endfirsthead
\toprule()
Puesto & Estudiante & Puntos Obtenidos & Puntos Adic. & Diploma \\
\midrule()
\endhead
1 & URIBE, JUAN DAVID & 38 & 3 & SI \\
2 & CARRILLO, MARÍA ALEJANDRA & 35 & 0 & SI \\
3 & ROJAS BARRAGÁN, SEBASTIAN & 34 & 2 & SI \\
4 & RUEDA LÓPEZ, CARLOS MARIO & 33 & 2 & SI \\
5 & RODRIGUEZ, MARIA PAULA & 33 & 1 & SI \\
6 & RUIZ ARAGÓN, RAÚL SEBASTIÁN & 33 & 1 & SI \\
7 & COCONUBO SANTAMARÍA, MARIA CAMILA & 32 & 2 & SI \\
8 & LANDEIRA QUEIROZ, MARIA LUISA & 32 & 2 & NO \\
9 & ROQUE MATTA, LAURA CAMILA & 32 & 2 & SI \\
10 & CASTRO TIRADO, DAVID & 31 & 2 & SI \\
11 & PINO ROZO, DAVID & 31 & 2 & SI \\
12 & PATIÑO LOZANO, ANA SOFÍA & 31 & 1 & SI \\
13 & ÁLVAREZ ÁLVAREZ, GABRIELA & 30 & 2 & SI \\
14 & TACHACK ECHEVERRY, LUCIA & 30 & 2 & SI \\
15 & OJEDA RODRIGUEZ, MARIANA & 30 & 1 & SI \\
16 & CARRASCAL DUARTE, SOFIA & 29 & 2 & SI \\
17 & GARCIA BETANCUR, TOMAS & 29 & 1 & SI \\
18 & PEÑA AMAYA, RICARDO ENRIQUE & 29 & 1 & SI \\
19 & ORTEGA MARTINEZ, MARIA JULIANA & 28 & 2 & SI \\
20 & PRIETO LÓPEZ, ISABELLA & 28 & 1 & SI \\
21 & RODRIGUEZ, JULIAN ANDRÉS & 28 & 1 & NO \\
22 & VARGAS VACCA, LUNA VALERIA & 28 & 1 & SI \\
23 & ACEVEDO DAZA, PAULA ANDREA & 28 & 0 & SI \\
24 & TORRES MEDINA, JUAN ESTEBAN & 27 & 2 & NO \\
25 & VEGA RODRÍGUEZ, JUAN DIEGO & 27 & 2 & NO \\
26 & CASTRO HERNANDEZ, RAFAEL DAVID & 27 & 1 & SI \\
27 & CUERVO, MARÍA JOSÉ & 27 & 1 & SI \\
\bottomrule()
\end{longtable}

Por otro lado, los estudiantes atendieron 16 cursos distribuídos como se
muestra en la Figura~\ref{fig-distribucion-asignaturas} . La mayoría de
estudiantes se inscribió en Español A: Literatura (Grupo 1) en el nivel
medio, mientras que la mayoría optó por tomar English B (Grupo 2) y
Global Politics en el nivel superior pero solo 4 se presentaron como
candidatos en la materia de History SL (Grupo 3)

En el caso de las ciencias (Grupo 4), los estudiantes debían escoger de
manera obligatoria entre física y química (aunque se permitió que un
estudiante tomara ambas\footnote{Sebastián Rojas Barragan, quien obtuvo
  34 puntos totales, atendiendo física y química en el nivel superior
  con notas de 6 y 4 respectivamente.}). 33 estudiantes tomaron química
(19 en nivel superior y 14 en el nivel medio), mientras que solo 21
tomaron física (12 en NS y 9 en NM).

Biología NS, Francés ab initio NM y Computer Science SL eran materias
electivas por lo que eran excluyentes entre ellas. 30 estudiantes
atendieron Francés ab initio, y 9 Computer Science, ambas ofrecidas solo
en nivel medio. Biología, que se ofreció como electiva (para aquellos
estudiantes interesados en tomar dos ciencias) presentó 12 candidatos.
Finalmente, en el caso de matemáticas, el GCB ofrece solo una de las
asignaturas ofrecidas por el Diploma en ambos niveles:
\textbf{Matemáticas - Análisis y Enfoques}. En el nivel superior se
presentaron 28 estudiantes, mientras que 25 lo hicieron en el nivel
medio.

\begin{figure}

{\centering \includegraphics{Resultados-DP-2023_pdf_files/figure-pdf/fig-distribucion-asignaturas-1.pdf}

}

\caption{\label{fig-distribucion-asignaturas}Alumnos inscritos por
asignatura}

\end{figure}

\hypertarget{resultados}{%
\subsection{2. Resultados}\label{resultados}}

Se presentó la siguiente distribución de resultados (
Figura~\ref{fig-distribución-resultados}). En general, la mayoría de
estudiantes que obtuvo el diploma, alcanzaron el rango de los 25 a los
30 puntos. Como se mencionó al inicio, el promedio de puntos obtenidos
por aquellos quienes obtuvieron el Diploma IB fue de 29 puntos, uno por
debajo del promedio obtenido por los estudiantes que obtuvieron su
diploma en la sesión de noviembre 2022.

La calificación promedio de alumnos del colegio que obtuvo el Diploma en
esta convocatoria fue de 4,60; 10 décimas por debajo de la calificación
promedio obtenida en noviembre de 2022 (4,7)

\begin{figure}

{\centering \includegraphics{Resultados-DP-2023_pdf_files/figure-pdf/fig-distribución-resultados-1.pdf}

}

\caption{\label{fig-distribución-resultados}Distribución de puntajes
convocatoria mayo 2023}

\end{figure}

Los promedios de los resultados obtenidos por asignatura en la sesión de
mayo 2023 se muestran en la Tabla~\ref{tbl-promedio} y la
Figura~\ref{fig-promedio-asignaturas} siguiente:

\hypertarget{tbl-promedio}{}
\setlength{\LTpost}{0mm}
\begin{longtable}{lrrr}
\caption{\label{tbl-promedio}Promedios GCB y Mundiales IB 2023 }\tabularnewline

\caption*{
{\large \textbf{Resultados por asignaturas Diploma IB}}
} \\ 
\toprule
 & \multicolumn{2}{c}{\textbf{Nota Promedio 2023}} &  \\ 
\cmidrule(lr){2-3}
Asignatura & GCB\textsuperscript{\textit{*}} & Mundial & GCB(2022) \\ 
\midrule
ESPAÑOL A NM & $5.18$ & $5.51$ & $5.20$ \\ 
ESPAÑOL A NS & $4.90$ & $4.72$ & $4.92$ \\ 
INGLÉS B NM & $5.88$ & $5.64$ & $6.47$ \\ 
INGLÉS B NS & $5.86$ & $5.70$ & $5.75$ \\ 
FRANCÉS AB NM & $4.80$ & $4.79$ & $4.71$ \\ 
POLÍTICA GLOBAL NM & $4.79$ & $4.82$ & $4.04$ \\ 
POLÍTICA GLOBAL NS & $4.68$ & $5.11$ & $5.03$ \\ 
HISTORIA NM & $3.50$ & $4.66$ & $4.40$ \\ 
FÍSICA NM & $3.65$ & $4.21$ & $3.00$ \\ 
FÍSICA NS & $4.17$ & $4.81$ & $3.68$ \\ 
QUÍMICA NM & $2.29$ & $4.10$ & $2.88$ \\ 
QUÍMICA NS & $2.53$ & $4.57$ & $2.92$ \\ 
BIOLOGÍA NS & $4.08$ & $4.40$ & $3.92$ \\ 
COMPUTER SCIENCE NM & $3.89$ & $3.94$ & - \\ 
MATEMÁTICAS NM & $3.36$ & $4.61$ & $3.78$ \\ 
MATEMÁTICAS NS & $3.43$ & $4.88$ & $4.21$ \\ 
\bottomrule
\end{longtable}
\begin{minipage}{\linewidth}
\textsuperscript{\textit{*}}Los resultados en cyan corresponden a las asignaturas que obtuvieron promedios por encima del promedio mundial. Los resultados en rosa establecen las asignaturas con menores resultados en la convocatoria\\
\end{minipage}

\begin{figure}

\begin{minipage}[t]{0.50\linewidth}

{\centering 

\raisebox{-\height}{

\includegraphics{Resultados-DP-2023_pdf_files/figure-pdf/fig-promedio-asignaturas-1.pdf}

}

}

\subcaption{\label{fig-promedio-asignaturas-1}Grupo 1 y Grupo 2: Español
A, Inglés B, Francés ab initio}
\end{minipage}%
%
\begin{minipage}[t]{0.50\linewidth}

{\centering 

\raisebox{-\height}{

\includegraphics{Resultados-DP-2023_pdf_files/figure-pdf/fig-promedio-asignaturas-2.pdf}

}

}

\subcaption{\label{fig-promedio-asignaturas-2}Grupo 3: Política Global,
Historia}
\end{minipage}%
\newline
\begin{minipage}[t]{0.50\linewidth}

{\centering 

\raisebox{-\height}{

\includegraphics{Resultados-DP-2023_pdf_files/figure-pdf/fig-promedio-asignaturas-3.pdf}

}

}

\subcaption{\label{fig-promedio-asignaturas-3}Grupo 4: Física, Química,
Biología y Computer Science}
\end{minipage}%
%
\begin{minipage}[t]{0.50\linewidth}

{\centering 

\raisebox{-\height}{

\includegraphics{Resultados-DP-2023_pdf_files/figure-pdf/fig-promedio-asignaturas-4.pdf}

}

}

\subcaption{\label{fig-promedio-asignaturas-4}Grupo 5: Matemáticas:
Análisis y Enfoques}
\end{minipage}%

\caption{\label{fig-promedio-asignaturas}Promedios GCB 2023, promedio
mundial y promedio GCB 2022}

\end{figure}

\hypertarget{comentarios-sobre-los-resultados-promedio}{%
\subsubsection{2.1 Comentarios sobre los resultados
promedio}\label{comentarios-sobre-los-resultados-promedio}}

De estos resultados, se tiene que las asignaturas del Grupo 1 (Español
A: Literatura) y el Grupo 2 (English B y Francés \emph{ab initio}) estan
en los niveles del promedio mundial. Tanto English B como Francés
\emph{ab initio}, así como Español A NS, obtuvieron promedios superiores
al promedio alcanzado mundialmente.

En el caso del Grupo 3, para Global Politics NM se observa un promedio
alcanzado significativamente equivalente al promedio mundial. Para el
nivel superior, el promedio del GCB es menor al obtenido de manera
global (4,68 frente a 5,11 en el mundo). En el caso de History NM, los
cuatro estudiantes que se presentaron como candidatos en esta asignatura
obtuvieron en promedio una nota un punto y medio por debajo (24,9\%) del
promedio mundial\footnote{De los cuatro estudiantes inscritos, dos
  (Maria Alejandra Carrillo y Maria Paula Rodriguez) obtuvieron una nota
  de 4.}.

Las ciencias en general tuvieron un desempeño por debajo de los
promedios mundiales. Computer Science SL (que presentó candidatos por
primera vez) y Biología NS fueron las asignaturas con los puntajes más
cercanos a los promedios mundiales (3,89 frente a 3,94 en el caso de
Computer Science y 4,08 frente a 4,4 en el caso de Biolgía). Se destaca
el resultado de Biología cuyos estudiantes obtuvieron en promedio una
nota de 4,1 (aprobatoria), superando al promedio obtenido en la
convocatoria de 2022 (3,92).

Para el caso de física, los estudiantes inscritos obtuvieron notas (en
ambos niveles) en promedio inferiores a los promedios alcanzados
mundialmente. En el caso del nivel superior, sin embargo, el promedio
obtenido aunque inferior al del mundo (4,87), es superior a 4 (4,17),
considerándose un promedio aprobatorio. Se destaca que para ambos
niveles, los resultados obtenidos en esta convocatoria fueron superiores
a los obtenidos por los estudiantes de la primera promoción de IB en
2022 ( Tabla~\ref{tbl-variacion-promedios} ).

Por otro lado, el caso de química presenta los mayores desafíos
pedagógicos y de evaluación. En esta convocatoria, los promedios
obtenidos por los estudiantes estuvieron muy por debajo de los promedios
mundiales, con casi dos puntos por debajo tanto en el nivel medio como
en el nivel superior. De igual manera, los resultados obtenidos en esta
asignatura durante 2023 fueron inferiores a los obtenidos en 2022, donde
también los resultados estuvieron por debajo de los promedios mundiales.

Finalmente, para el caso de Matemáticas: Análisis y Enfoques, se tiene
que los promedios alcanzados por los estudiantes durante 2023 fueron
inferiores a los promedios mundiales en al menos un punto y medio para
ambos niveles. De la misma manera, los resultados obtenidos en esta
convocatoria fueron inferiores a los obtenidos en la convocatoria de
2022 ( Tabla~\ref{tbl-variacion-promedios} ).

\hypertarget{tbl-variacion-promedios}{}
\begin{longtable}{lrrr}
\caption{\label{tbl-variacion-promedios}Variación promedios obtenidos por el GCB 2022-2023 }\tabularnewline

\toprule
Asignatura & GCB 2023 & GCB 2022 & Variación Porcentual \\ 
\midrule
ESPAÑOL A NM & 5.18 & 5.20 & $-0.38\%$ \\ 
ESPAÑOL A NS & 4.90 & 4.92 & $-0.41\%$ \\ 
INGLÉS B NM & 5.88 & 6.47 & $-9.12\%$ \\ 
INGLÉS B NS & 5.86 & 5.75 & $1.91\%$ \\ 
FRANCÉS AB NM & 4.80 & 4.71 & $1.91\%$ \\ 
POLÍTICA GLOBAL NM & 4.79 & 4.04 & $18.56\%$ \\ 
POLÍTICA GLOBAL NS & 4.68 & 5.03 & $-6.96\%$ \\ 
HISTORIA NM & 3.50 & 4.40 & $-20.45\%$ \\ 
FÍSICA NM & 3.65 & 3.00 & $21.67\%$ \\ 
FÍSICA NS & 4.17 & 3.68 & $13.32\%$ \\ 
QUÍMICA NM & 2.29 & 2.88 & $-20.49\%$ \\ 
QUÍMICA NS & 2.53 & 2.92 & $-13.36\%$ \\ 
BIOLOGÍA NS & 4.08 & 3.92 & $4.08\%$ \\ 
COMPUTER SCIENCE NM & 3.89 & — & — \\ 
MATEMÁTICAS NM & 3.36 & 3.78 & $-11.11\%$ \\ 
MATEMÁTICAS NS & 3.43 & 4.21 & $-18.53\%$ \\ 
\bottomrule
\end{longtable}

\hypertarget{aprobaciuxf3n-por-asignatura}{%
\subsubsection{2.2 Aprobación por
asignatura}\label{aprobaciuxf3n-por-asignatura}}

En esta convocatoria se presentarón los siguientes porcentajes de
aprobación\footnote{Se establece como \textbf{aprobación}, el porcentaje
  de estudiantes que obtuvo una nota de 4 o más en el resultado final de
  la asignatura.} ( Tabla~\ref{tbl-porcentajes} ). Como es evidente, las
asignaturas con menores promedios obtenidos fueron a su vez las
asignaturas con menor porcentaje de aprobación. \textbf{Química} fue la
asignatura con menores promedios y con el menor número de estudiantes
que obtuvo una nota igual o superior a 4. En el caso del nivel medio,
todos los estudiantes inscritos (15) obtuvieron notas de 3 o menos. En
el caso del nivel superior, de los 20 estudiantes inscritos, solo dos
obtuvieron una nota igual o superior a 4.

Se destacan las asignaturas de \textbf{Español A, English B, Global
Politics} y \textbf{Francés ab initio}, en donde la mayoría de
estudiantes inscritos obtuvo notas superiores o iguales a 4.

\hypertarget{tbl-porcentajes}{}
\begin{longtable}{lrr}
\caption{\label{tbl-porcentajes}Porcentajes de aprobación por asignatura - Mayo 2023 }\tabularnewline

\caption*{
{\large \textbf{Resultados por asignaturas Diploma IB}}
} \\ 
\toprule
Asignatura & Promedio GCB & Porcentaje Aprobación \\ 
\midrule
BIOLOGÍA NS & $4.05$ & $69\%$ \\ 
COMPUTER SCIENCE SL & $3.84$ & $50\%$ \\ 
ENGLISH B SL & $5.84$ & $100\%$ \\ 
ENGLISH B HL & $5.86$ & $100\%$ \\ 
ESPAÑOL NM & $5.18$ & $97\%$ \\ 
ESPAÑOL NS & $4.89$ & $100\%$ \\ 
FRANCÉS AB NM & $4.79$ & $94\%$ \\ 
FÍSICA NM & $3.50$ & $40\%$ \\ 
FÍSICA NS & $4.10$ & $54\%$ \\ 
GLOBAL POLITICS SL & $4.76$ & $100\%$ \\ 
GLOBAL POLITICS HL & $4.67$ & $100\%$ \\ 
HISTORY SL & $3.27$ & $40\%$ \\ 
MATEMÁTICAS NM & $3.35$ & $31\%$ \\ 
MATEMÁTICAS NS & $3.42$ & $38\%$ \\ 
QUÍMICA NM & $2.27$ & $0\%$ \\ 
QUMÍMICA NS & $2.52$ & $10\%$ \\ 
\bottomrule
\end{longtable}

\begin{tcolorbox}[enhanced jigsaw, colbacktitle=quarto-callout-important-color!10!white, toprule=.15mm, colback=white, toptitle=1mm, leftrule=.75mm, rightrule=.15mm, left=2mm, arc=.35mm, colframe=quarto-callout-important-color-frame, breakable, bottomtitle=1mm, titlerule=0mm, opacityback=0, title=\textcolor{quarto-callout-important-color}{\faExclamation}\hspace{0.5em}{Importante}, coltitle=black, opacitybacktitle=0.6, bottomrule=.15mm]
Es esencial que se haga una revisión a fondo de la \textbf{planeación} y
\textbf{desarrollo} de las asignaturas de ciencias, especialmente en los
casos de \textbf{física} y \textbf{química}, así como evaluar la manera
en que debe mejorarse tanto la enseñanza como el aprendizaje de los
cursos de \textbf{matemáticas}, cuyos resultados son bajos.
\end{tcolorbox}

\hypertarget{distribuciuxf3n-de-calificaciuxf3n-por-grupo-de-asignaturas}{%
\subsubsection{2.3 Distribución de calificación por grupo de
asignaturas}\label{distribuciuxf3n-de-calificaciuxf3n-por-grupo-de-asignaturas}}

En la figura siguiente (
Figura~\ref{fig-distribucion-calificaciones-asignatura} ) puede
observarse la distribución de calificaciones para cada uno de los grupos
de asignaturas. Como se observa, en el Grupo 1 las mejores
calificaciones se obtuvieron en Español A con 41 estudiantes de 53
(repartidos en NS y NM) en los nivel de calificación 5-7. Se destaca que
todos los estudiantes estuvieron en este mismo rango de calificación en
la asignatura English B. En el caso de Francés ab initio, 19 de los 30
estudiantes que atendieron la clase estuvieron también en los niveles 5
a 7.

En el caso del Grupo 3, las mejores calificaciones se obtuvieron en
Política Global, con 31 estudiantes en el rango 5-6, mientras que en
Historia con cuatro estudiantes inscritos, dos estudiantes obtuvieron
una nota de 3 y los otros dos, una nota de 4.

Como se ha mencionado anteriormente, el Grupo 4: Ciencias, sigue
presentando los mayores desafíos. Ningún estudiante obtuvo una
calificación de 7 en alguna de las asignaturas del grupo. La mayoría
obtuvo notas en los niveles 2 y 3. Solo en el caso de química, por
ejemplo, 31 de los 33 estudiantes obtuvieron notas entre 1 y 3. En el
caso de Biología en cambio, 9 de los 12 estudiantes inscritos obtuvieron
notas de 4 y 5, lo que influyó en el promedio aprobatorio obtenido.

Finalmente, en el caso del Grupo 5: Mátemáticas, 34 de los 53
estudiantes obtuvo una nota de 2 o 3 en su asignatura, y 19 estuvieron
en los niveles 4 o 5, siendo esta última, la nota más alta obtenida en
la asignatura (por 4 estudiantes de nivel superior y 2 de nivel medio)

\begin{figure}

\begin{minipage}[t]{0.50\linewidth}

{\centering 

\raisebox{-\height}{

\includegraphics{Resultados-DP-2023_pdf_files/figure-pdf/fig-distribucion-calificaciones-asignatura-1.pdf}

}

}

\subcaption{\label{fig-distribucion-calificaciones-asignatura-1}Grupo1 y
Grupo 2: Español A, Inglés B, Francés ab initio}
\end{minipage}%
%
\begin{minipage}[t]{0.50\linewidth}

{\centering 

\raisebox{-\height}{

\includegraphics{Resultados-DP-2023_pdf_files/figure-pdf/fig-distribucion-calificaciones-asignatura-2.pdf}

}

}

\subcaption{\label{fig-distribucion-calificaciones-asignatura-2}Grupo 3:
Política Global, Historia}
\end{minipage}%
\newline
\begin{minipage}[t]{0.50\linewidth}

{\centering 

\raisebox{-\height}{

\includegraphics{Resultados-DP-2023_pdf_files/figure-pdf/fig-distribucion-calificaciones-asignatura-3.pdf}

}

}

\subcaption{\label{fig-distribucion-calificaciones-asignatura-3}Grupo 4:
Física, Química, Biología y Computer Science}
\end{minipage}%
%
\begin{minipage}[t]{0.50\linewidth}

{\centering 

\raisebox{-\height}{

\includegraphics{Resultados-DP-2023_pdf_files/figure-pdf/fig-distribucion-calificaciones-asignatura-4.pdf}

}

}

\subcaption{\label{fig-distribucion-calificaciones-asignatura-4}Grupo 5:
Matemáticas}
\end{minipage}%

\caption{\label{fig-distribucion-calificaciones-asignatura}Distribución
de calificaciones por asignatura}

\end{figure}

\hypertarget{distribuciuxf3n-de-puntajes-totales-por-asignatura}{%
\subsubsection{2.4 Distribución de puntajes totales por
asignatura}\label{distribuciuxf3n-de-puntajes-totales-por-asignatura}}

En la Tabla~\ref{tbl-puntos-promedio} y en la
Figura~\ref{fig-distribucion-puntos-asignatura} se muestra información
más detallada sobre los resultados de la convocatoria de mayo 2023 para
las diferentes asignaturas y sus niveles, incluyendo el promedio de
puntuación y la desviación estándar de las notas en cada asignatura y
nivel.

La puntuación promedio indica el valor medio de las puntuaciones
obtenidas por los estudiantes en cada asignatura y nivel\footnote{El
  puntaje total (en una escala de 0 a 100) es la suma de puntos
  obtenidos por cada estudiante en cada uno de los componentes de
  evaluación en su asignatura (pruebas externas, trabajos internos,
  etc), ponderados por el peso que tiene cada uno de ellos en su
  calificación final.''}. Por ejemplo, en la asignatura \textbf{English
B NS} el promedio de puntuación es de 79.6, lo que se tradujo que en
promedio los estudiantes obtuvieran una calificación en los niveles más
altos de la escala.

La desviación estándar establece la variabilidad de las notas en cada
asignatura y nivel. Una desviación estándar alta, como en
\textbf{Matemáticas: Análisis y Enfoques NS} (8.78), indica que las
notas de los estudiantes están más dispersas en esta asignatura y nivel,
lo que significa que hay una mayor variabilidad en el rendimiento
académico de los estudiantes. Por el contrario, una desviación estándar
baja, como en \textbf{Química NM} (4.70), sugiere que las notas están
más concentradas alrededor del promedio, lo que indica una menor
variabilidad en el rendimiento académico de los alumnos.

Los mayores promedios de puntajes totales se encuentran en
\textbf{English B NS} (79.6) y \textbf{English B NM} (78.1), lo que
indica que los estudiantes, así como en la convocatoria de 2022, tienden
a obtener calificaciones altas en esta asignatura independientemente del
nivel en el que se inscriban. Por otro lado, los menores promedios se
encuentran en \textbf{Matemáticas NM} (26.9) y \textbf{Química NM}
(25.2).

Las mayores dispersiones en el puntaje obtenido se encuentró en
\textbf{Física NS} (12.8) y \textbf{Español A NM} (12.0), lo que indica
que las notas en estas asignaturas y niveles mostraron una mayor
variabilidad en el rendimiento de los estudiantes, o en últimas, que el
desempeño de los alumnos tiene diferencias muy altas entre ellos. Las
menores desviaciones estándar se encuentran en \textbf{Química NM}
(4.70) y \textbf{Política Global NM} (5.90), lo que indica un desempeño
mucho más estandarizado por parte de los estudiantes: tienen similares
niveles de comprensión y de dificultad.

\begin{figure}

\begin{minipage}[t]{0.50\linewidth}

{\centering 

\raisebox{-\height}{

\includegraphics{Resultados-DP-2023_pdf_files/figure-pdf/fig-distribucion-puntos-asignatura-1.pdf}

}

}

\subcaption{\label{fig-distribucion-puntos-asignatura-1}Grupo1 y Grupo
2: Estudios de Lengua y Literatura y Adquisición de Lenguas}
\end{minipage}%
%
\begin{minipage}[t]{0.50\linewidth}

{\centering 

\raisebox{-\height}{

\includegraphics{Resultados-DP-2023_pdf_files/figure-pdf/fig-distribucion-puntos-asignatura-2.pdf}

}

}

\subcaption{\label{fig-distribucion-puntos-asignatura-2}Grupo 3:
Individuos y Sociedades}
\end{minipage}%
\newline
\begin{minipage}[t]{0.50\linewidth}

{\centering 

\raisebox{-\height}{

\includegraphics{Resultados-DP-2023_pdf_files/figure-pdf/fig-distribucion-puntos-asignatura-3.pdf}

}

}

\subcaption{\label{fig-distribucion-puntos-asignatura-3}Grupo 4:
Ciencias}
\end{minipage}%
%
\begin{minipage}[t]{0.50\linewidth}

{\centering 

\raisebox{-\height}{

\includegraphics{Resultados-DP-2023_pdf_files/figure-pdf/fig-distribucion-puntos-asignatura-4.pdf}

}

}

\subcaption{\label{fig-distribucion-puntos-asignatura-4}Grupo 5:
Matemáticas}
\end{minipage}%

\caption{\label{fig-distribucion-puntos-asignatura}Distribución de
puntos obtenidos por asignatura}

\end{figure}

\hypertarget{tbl-puntos-promedio}{}
\setlength{\LTpost}{0mm}
\begin{longtable}{llll}
\caption{\label{tbl-puntos-promedio}Puntuación promedio por asignatura - Mayo 2023 }\tabularnewline

\toprule
Asignatura & Nivel & Puntuación Promedio\textsuperscript{\textit{*}} & Desviación estándar \\ 
\midrule
BIOLOGÍA & NS & $42.75$ & $9.78$ \\ 
ESPAÑOL A: Literatura & NS & $58.50$ & $6.22$ \\ 
ESPAÑOL A: Literatura & NM & $63.39$ & $11.99$ \\ 
FRANCÉS AB INITIO & NM & $59.03$ & $10.97$ \\ 
FÍSICA & NS & $43.83$ & $12.77$ \\ 
FÍSICA & NM & $31.89$ & $7.98$ \\ 
HISTORIA & NM & $37.50$ & $8.94$ \\ 
COMPUTER SCIENCE & NM & $46.33$ & $9.11$ \\ 
INGLÉS B & NS & $79.62$ & $6.72$ \\ 
INGLÉS B & NM & $78.06$ & $6.85$ \\ 
MATEMÁTICAS: ANÁLISIS Y ENFOQUES & NS & $29.79$ & $8.78$ \\ 
MATEMÁTICAS: ANÁLISIS Y ENFOQUES & NM & $26.88$ & $7.79$ \\ 
POLÍTICA GLOBAL & NS & $47.90$ & $6.93$ \\ 
POLÍTICA GLOBAL & NM & $45.79$ & $5.90$ \\ 
QUÍMICA & NS & $26.47$ & $8.58$ \\ 
QUÍMICA & NM & $25.21$ & $4.70$ \\ 
\bottomrule
\end{longtable}
\begin{minipage}{\linewidth}
\textsuperscript{\textit{*}}Los resultados en cyan corresponden a las asignaturas que obtuvieron los puntajes más altos durante la convocatiria\\
\end{minipage}

\hypertarget{resultados-monografuxeda-y-tdc}{%
\subsubsection{2.5 Resultados Monografía y
TdC}\label{resultados-monografuxeda-y-tdc}}

\hypertarget{monografuxeda}{%
\paragraph{2.5.1 Monografía}\label{monografuxeda}}

Se presentaron 53 trabajos de monografía que se distribuyeron como se
ilustra en la Tabla~\ref{tbl-monografia} . En esta convocatoria, el
mayor número de trabajos se realizó en las asignaturas del Grupo 3:
Individuos y Sociedades. Política Global presentó 14 trabajos e Historia
2 más. Español sin embargo presentó ella sola 12 trabajos de monografía.

A diferencia de la primera convocatoria, en 2023 solo se realizaron 8
trabajos en Ciencias (Biología, Química y Computer Science), pero se
dobló el número de trabajos realizados en Matemáticas (6 monografías,
frente a 3 de la primera convocatoria.)

Se destaca que de las 53 monografías, 16 de ellas (30\%) se realizaron
en inglés, un número igual de trabajos a los que se presentaron en este
idioma en 2022. En esta ocasión, se presentaron en inglés 11 monografías
en la materia \textbf{English B}, 3 en \textbf{Computer Science}, y 2 en
\textbf{Política Global.}

\hypertarget{tbl-monografia}{}
\begin{longtable}{llc}
\caption{\label{tbl-monografia}Distribución de Monografías 2023 }\tabularnewline

\toprule
Asignatura & Idioma & Monografíasrealizadas \\ 
\midrule
ESPAÑOL A & ESPAÑOL & $12$ \\ 
POLÍTICA GLOBAL & ESPAÑOL & $12$ \\ 
INGLÉS B & INGLÉS & $11$ \\ 
MATEMÁTICAS & ESPAÑOL & $6$ \\ 
BIOLOGÍA & ESPAÑOL & $4$ \\ 
COMPUTER SCIENCE & INGLÉS & $3$ \\ 
HISTORIA & ESPAÑOL & $2$ \\ 
POLÍTICA GLOBAL & INGLÉS & $2$ \\ 
QUÍMICA & ESPAÑOL & $1$ \\ 
\bottomrule
\end{longtable}

La mayoría de los trabajos de monografía (39 de 53) estuvieron en los
niveles B y C de calificación lo que corresponde a niveles
satisfactorios para estos trabajos ( Figura~\ref{fig-monografias} ). Sin
embargo, de las monografías presentadas, 13 de ellas tuvieron
calificación D, un 18\% más que las que obtuvieron la misma calificación
en la convocatoria de noviembre 2022. Mientras que en 2022, seis
estudiantes tuvieron la mayor calificación (A) en su trabajo, para 2023
solo un candidato alcanzó esta distinción (ver
Tabla~\ref{tbl-mejormonografia-variacion}). Ninguno de nuestros
estudiantes, al igual que en 2022, obtuvo la menor calificación en este
componente (E), que excluye a los alumnos de la obtención del Diploma.

La Tabla~\ref{tbl-mejormonografia-variacion} muestra también los datos
relacionados con calificaciones de la monografia para 2022 y 2023, junto
con la variación entre estos dos años.

\begin{figure}

\begin{minipage}[t]{0.50\linewidth}

{\centering 

\raisebox{-\height}{

\includegraphics{Resultados-DP-2023_pdf_files/figure-pdf/fig-monografias-1.pdf}

}

}

\subcaption{\label{fig-monografias-1}Distribución calificación
Monografías}
\end{minipage}%
%
\begin{minipage}[t]{0.50\linewidth}

{\centering 

\raisebox{-\height}{

\includegraphics{Resultados-DP-2023_pdf_files/figure-pdf/fig-monografias-2.pdf}

}

}

\subcaption{\label{fig-monografias-2}Puntos ajustados de Monografía por
asignaturas}
\end{minipage}%

\caption{\label{fig-monografias}Calificación Monografías - Mayo 2023}

\end{figure}

\begin{table}

\caption{\label{tbl-mejormonografia-variacion}Mejores trabajos de
Monografía 2023 - Variación de resultados Monografía
2022-2023}\begin{minipage}[t]{\linewidth}
\subcaption{\label{tbl-mejormonografia-variacion-1}Mejores monografías 2023 }

{\centering 

\begin{longtable}{lllll}
\tabularnewline

\caption*{
{\large \textbf{Los 10 mejores trabajos de Monografía - Mayo 2023}}
} \\ 
\toprule
Asignatura & Idioma & Estudiante & Puntuación & Calificación \\ 
\midrule
ESPAÑOL A & ESPAÑOL & PINO ROZO, DAVID & 30 & A \\ 
ESPAÑOL A & ESPAÑOL & CASTRO TIRADO, DAVID & 26 & B \\ 
ESPAÑOL A & ESPAÑOL & COCONUBO SANTAMARÍA, MARIA CAMILA & 26 & B \\ 
ESPAÑOL A & ESPAÑOL & LANDEIRA QUEIROZ, MARIA LUISA & 26 & B \\ 
ESPAÑOL A & ESPAÑOL & RAMIREZ ZORRO, ALEJANDRO & 26 & B \\ 
ESPAÑOL A & ESPAÑOL & TACHACK ECHEVERRY, LUCIA & 26 & B \\ 
ESPAÑOL A & ESPAÑOL & VEGA RODRÍGUEZ, JUAN DIEGO & 26 & B \\ 
INGLÉS B & INGLÉS & GÓMEZ HERRERA, MARIA CAMILA & 26 & B \\ 
BIOLOGÍA & ESPAÑOL & FERNANDEZ, MARIA CAMILA & 24 & B \\ 
BIOLOGÍA & ESPAÑOL & ROQUE MATTA, LAURA CAMILA & 24 & B \\ 
\bottomrule
\end{longtable}

}

\end{minipage}%
\newline
\begin{minipage}[t]{\linewidth}
\subcaption{\label{tbl-mejormonografia-variacion-2}Variación porcentual de calificación de la monografía }

{\centering 

\begin{longtable}{llll}
\tabularnewline

\toprule
Calificacion & Nov 2022 & May 2023 & Variación \\ 
\midrule
A & $5.3\%$ & $0.9\%$ & $-83.3\%$ \\ 
B & $8.8\%$ & $14.9\%$ & $70.0\%$ \\ 
C & $29.8\%$ & $19.3\%$ & $-35.3\%$ \\ 
D & $9.6\%$ & $11.4\%$ & $18.2\%$ \\ 
E & $0.0\%$ & $0.0\%$ & ---- \\ 
\bottomrule
\end{longtable}

}

\end{minipage}%

\end{table}

Se presentó un aumento significativo en la obtención de la calificación
B , con una variación positiva del 70\%. De la misma manera, la
calificación C muestra una disminución en el porcentaje de trabajos
calificados entre las dos convocatorias, con una variación negativa del
35.3\%. Sin embargo, la calificación A muestra una disminución aún más
pronunciada en el porcentaje de trabajos entre las dos convocatorias,
con una variación negativa del 83.3\%. Mientras que en 2022, seis
estudiantes (de un total de 61) alcanzaron la mejor calificación, en
2023 solo uno de 53 la alcanzó.

Finalmente, se evidencia un pequeño aumento en el porcentaje de
observaciones para la calificación D entre los años 2022 y 2023, con una
variación positiva del 18.2\%.

\begin{tcolorbox}[enhanced jigsaw, colbacktitle=quarto-callout-important-color!10!white, toprule=.15mm, colback=white, toptitle=1mm, leftrule=.75mm, rightrule=.15mm, left=2mm, arc=.35mm, colframe=quarto-callout-important-color-frame, breakable, bottomtitle=1mm, titlerule=0mm, opacityback=0, title=\textcolor{quarto-callout-important-color}{\faExclamation}\hspace{0.5em}{Importante}, coltitle=black, opacitybacktitle=0.6, bottomrule=.15mm]
Es \textbf{fundamental} trabajar en el acompañamiento a los estudiantes
del Diploma en la elaboración de su tabajo de monografía. La
\textbf{comprensión}, no solo de los criterios de evaluación, sino de
los elementos que se requieren para el desarrollo del trabajo, implica
una capacitación constante por parte de los profesores y supervisores.
\end{tcolorbox}

\hypertarget{teoruxeda-del-conocimiento}{%
\paragraph{2.5.2 Teoría del
Conocimiento}\label{teoruxeda-del-conocimiento}}

En el caso de TdC, el desempeño de los estudiantes fue similar a los
reultados de 2022, con algunas mejoras puntuales en la banda de
calificación ( Figura~\ref{fig-calificacion-tdc} ). Al igual que en
2022, seis estudiantes obtuvieron las calificaciones más altas del curso
(A y B). No obstante, en 2023, hubo una mayor proporción de estudiantes
que alcanzaron el nivel C de calificación (un aumento del 17\% de
estudiantes), lo que en últimas significó una reducción del porcentaje
de estudiantes que obtuvieron los niveles más bajos de evaluación (D y
E). Para esta convocatoria, la proporción de estudiantes que obtuvo
calificación D bajó del 22 al 12 por ciento; y en esta ocasión ningun
estudiante obtuvo la menor calificación (E), que es excluyente de
obtención del programa ( ver Tabla~\ref{tbl-variacion-tdc} ).

\begin{figure}

{\centering \includegraphics{Resultados-DP-2023_pdf_files/figure-pdf/fig-calificacion-tdc-1.pdf}

}

\caption{\label{fig-calificacion-tdc}Calificaciones finales TdC - Mayo
2023}

\end{figure}

\hypertarget{tbl-variacion-tdc}{}
\begin{longtable}{llll}
\caption{\label{tbl-variacion-tdc}Variación resultados TdC 2022 -2023 }\tabularnewline

\toprule
Calificacion & Nov 2022 & May 2023 & Variación \\ 
\midrule
A & $0.9\%$ & $0.9\%$ & $0.0\%$ \\ 
B & $4.4\%$ & $4.4\%$ & $0.0\%$ \\ 
C & $24.6\%$ & $28.9\%$ & $17.9\%$ \\ 
D & $22.8\%$ & $12.3\%$ & $-46.2\%$ \\ 
E & $0.9\%$ & $0.0\%$ & $-100.0\%$ \\ 
\bottomrule
\end{longtable}

\hypertarget{comparaciuxf3n-notas-predichas-y-notas-obtenidas-por-asignatura}{%
\subsubsection{2.6 Comparación notas predichas y notas obtenidas por
asignatura}\label{comparaciuxf3n-notas-predichas-y-notas-obtenidas-por-asignatura}}

Finalmente en este reporte, se presenta la comparación de las notas
obtenidas por los estudiantes en cada asignatura y las notas predichas
por los profesores, de la misma manera, se presenta la distribución de
las calificaciones moderadas (otorgadas por examinadores externos) par
los trabajos internos en cada asignatura.

La comparación entre notas previstas y notas obtenidas es importante
porque establece qué tan precisos son nuestros maestros en establecer
los niveles de alcance de sus estudiantes, y que tanto pueden determinar
-dados los descriptores de calificación de cada asignatura- cuales son
las habilidades que los estudiantes pueden lograr en el transcurso del
programa. Se establece en la Figura~\ref{fig-prevista-obtenida} la nota
prevista y la nota obtenida por asignatura por cada estudiante. Si la
nota prevista y la obtenida fueron equivalentes (precisión total por
parte del profesor de la asignatura) las notas estarán sobre la línea
diagonal. Si la nota obtenida es mayor que la nota predicha (predicción
por debajo de la nota final), los alumnos se encontrarán por encima de
la línea diagonal, y sucederá lo contrario si la nota obtenida es
inferior a la nota prevista.

Las notas finales de asignatura son variables discretas (niveles de
calificación de 1 a 7 que describen cada uno de ellos un conjunto de
atributos alcanzados por los estudiantes. Ver
\href{/Users/JorgeEd/OneDrive\%20-\%20Gimnasio\%20Colombo\%20Británico/Descriptores\%20de\%20calificaciones\%20finales\%20del\%20Programa\%20del\%20Diploma.pdf}{Descriptores
de Calificación Programa del Diploma} ), por lo que para mejorar la
visualización se agrega un ruido aleatorio a la nota de cada estudiante
con el fin de separar los resultados que se solapan entre sí.

\begin{figure}

\begin{minipage}[t]{0.50\linewidth}

{\centering 

\raisebox{-\height}{

\includegraphics{Resultados-DP-2023_pdf_files/figure-pdf/fig-prevista-obtenida-1.pdf}

}

}

\subcaption{\label{fig-prevista-obtenida-1}Grupo 1: Estudios de Lengua y
Literatura}
\end{minipage}%
%
\begin{minipage}[t]{0.50\linewidth}

{\centering 

\raisebox{-\height}{

\includegraphics{Resultados-DP-2023_pdf_files/figure-pdf/fig-prevista-obtenida-2.pdf}

}

}

\subcaption{\label{fig-prevista-obtenida-2}Grupo 2: Adquisición de
Lenguas}
\end{minipage}%
\newline
\begin{minipage}[t]{0.50\linewidth}

{\centering 

\raisebox{-\height}{

\includegraphics{Resultados-DP-2023_pdf_files/figure-pdf/fig-prevista-obtenida-3.pdf}

}

}

\subcaption{\label{fig-prevista-obtenida-3}Grupo 3: Individuos y
Sociedades}
\end{minipage}%
%
\begin{minipage}[t]{0.50\linewidth}

{\centering 

\raisebox{-\height}{

\includegraphics{Resultados-DP-2023_pdf_files/figure-pdf/fig-prevista-obtenida-4.pdf}

}

}

\subcaption{\label{fig-prevista-obtenida-4}Grupo 4: Ciencias: Física y
Química}
\end{minipage}%
\newline
\begin{minipage}[t]{0.50\linewidth}

{\centering 

\raisebox{-\height}{

\includegraphics{Resultados-DP-2023_pdf_files/figure-pdf/fig-prevista-obtenida-5.pdf}

}

}

\subcaption{\label{fig-prevista-obtenida-5}Grupo 4: Ciencias: Biología y
Computer Science}
\end{minipage}%
%
\begin{minipage}[t]{0.50\linewidth}

{\centering 

\raisebox{-\height}{

\includegraphics{Resultados-DP-2023_pdf_files/figure-pdf/fig-prevista-obtenida-6.pdf}

}

}

\subcaption{\label{fig-prevista-obtenida-6}Grupo 5: Matemáticas}
\end{minipage}%

\caption{\label{fig-prevista-obtenida}Comparación nota prevista vs.~nota
obtenida Diploma IB 2023}

\end{figure}

Se observa que para la convocatoria de 2023 las asignaturas de
Matemáticas y Física fueron en las que hubo una mayor precisión en la
calificación prevista frente a la obtenida por los estudiantes (
Figura~\ref{fig-diferencia-previstaobtenida} ). Sin embargo, cabe
destacar que al igual que en 2022, en la mayoría de los casos, el
promedio de las diferencias entre estas calificaciones es inferior a un
punto, lo cual es una variación considerada como aceptable y que denota
una comprensión de los elementos centrales de los descriptores de
calificación del IB, así como un seguimiento al proceso de aprendizaje
de los estudiantes que le permite a los profesores dar cuenta de los
alcances efectivos de los alumnos en cáda una de las clases.

En el caso de Español A: Literatura y English B, las diferencias
muestran que en promedio los profesores predijeron que sus estudiantes
estarían en promedio en algo menos de un punto por debajo de la nota
obtenida al final por los estudiantes. En el caso de Política Global,
ocurre el caso contrario, donde los estudiantes obtuvieron en promedio
medio punto por debajo de lo que se predijo que obtendrían.

Para esta convocatoria se encuentra que en el caso de Química, los
estudiantes obtuvieron en promedio un punto completo por debajo de la
nota prevista, al igual que en Historia, en la que los cuatro
estudiantes que se presentaron como candidatos obtuvieron en promedio un
punto y medio menos de lo que se havia previsto que obtuvieran.

\begin{figure}

{\centering \includegraphics{Resultados-DP-2023_pdf_files/figure-pdf/fig-diferencia-previstaobtenida-1.pdf}

}

\caption{\label{fig-diferencia-previstaobtenida}Diferencias promedio
entre nota obtenida y nota prevista por asignatura - Mayo 2023}

\end{figure}

\begin{tcolorbox}[enhanced jigsaw, colbacktitle=quarto-callout-tip-color!10!white, toprule=.15mm, colback=white, toptitle=1mm, leftrule=.75mm, rightrule=.15mm, left=2mm, arc=.35mm, colframe=quarto-callout-tip-color-frame, breakable, bottomtitle=1mm, titlerule=0mm, opacityback=0, title=\textcolor{quarto-callout-tip-color}{\faLightbulb}\hspace{0.5em}{Tip}, coltitle=black, opacitybacktitle=0.6, bottomrule=.15mm]
Es importante que los departamentos utilicen los atributos establecidos
en los
\href{/Users/JorgeEd/OneDrive\%20-\%20Gimnasio\%20Colombo\%20Británico/Descriptores\%20de\%20calificaciones\%20finales\%20del\%20Programa\%20del\%20Diploma.pdf}{Descriptores
de Calificación Programa del Diploma} para el diseño de desempeños e
instrumentos de evaluación. Los descriptores indican lo que se espera
que el estudiante sea capaz de alcanzar a lo largo del Programa. La
lectura y análisis de estos descriptores con los estudiantes, más el
ejemplo pedadógico de cómo estos atributos se transforman en elementos
prácticos dentro del salón de clase, permite una \textbf{comprensión}
mucho más efectiva de la conexión entre el currículo enseñado y el
currículo evaluado del Diploma.
\end{tcolorbox}

\hypertarget{resultados-trabajos-internos.}{%
\paragraph{2.6.2 Resultados trabajos
internos.}\label{resultados-trabajos-internos.}}

En la Figura~\ref{fig-trabajosinternos} se encuentra la distribución de
puntos (marcas) obtenidas en los \textbf{trabajos internos} realizados
en cada una de las asignaturas de la convocatoria de 2023. Es importante
recordar que los trabajos internos son realizados durante el desarrollo
del programa y son \textbf{calificados} por los profesores del colegio
pero \textbf{moderados} externamente.

Esto quiere decir que examinadores externos evaluan una \textbf{muestra
aleatoria} de los trabajos y con base en una evaluación estandarizada
realizada por un equipo de calificación (Ver
\href{https://gcbedu-my.sharepoint.com/:w:/g/personal/jorge_baquero_gcb_edu_co/ERPkujLt3cxOmSvqGgB_nZkBO3CTetFTRnR0So4bU0VZAw?e=a7stXi}{T}\href{https://gcbedu-my.sharepoint.com/:b:/g/personal/jorge_baquero_gcb_edu_co/EVuAFyiF1sxKgPkM42pP3coBNUqCyVD0jUB2mELz2BWs0w?e=FBcclx}{rabajos
Internos en el IB}).

\begin{figure}

\begin{minipage}[t]{0.50\linewidth}

{\centering 

\raisebox{-\height}{

\includegraphics{Resultados-DP-2023_pdf_files/figure-pdf/fig-trabajosinternos-1.pdf}

}

}

\subcaption{\label{fig-trabajosinternos-1}Grupo 1: Estudios de Lengua y
Literatura}
\end{minipage}%
%
\begin{minipage}[t]{0.50\linewidth}

{\centering 

\raisebox{-\height}{

\includegraphics{Resultados-DP-2023_pdf_files/figure-pdf/fig-trabajosinternos-2.pdf}

}

}

\subcaption{\label{fig-trabajosinternos-2}Grupo 2: Adquisición de
Lenguas}
\end{minipage}%
\newline
\begin{minipage}[t]{0.50\linewidth}

{\centering 

\raisebox{-\height}{

\includegraphics{Resultados-DP-2023_pdf_files/figure-pdf/fig-trabajosinternos-3.pdf}

}

}

\subcaption{\label{fig-trabajosinternos-3}Grupo 3: Individuos y
Sociedades}
\end{minipage}%
%
\begin{minipage}[t]{0.50\linewidth}

{\centering 

\raisebox{-\height}{

\includegraphics{Resultados-DP-2023_pdf_files/figure-pdf/fig-trabajosinternos-4.pdf}

}

}

\subcaption{\label{fig-trabajosinternos-4}Grupos 4 y 5: Ciencias y
Matemáticas}
\end{minipage}%

\caption{\label{fig-trabajosinternos}Distribución de calificación de
trabajos internos por asignaturas}

\end{figure}

En general, se observa que en la mayoría de las asignaturas, los
trabajos internos estuvieron por debajo de los puntos máximos en cada
una de ellas. Se destaca el caso de \textbf{English B,} en donde la
mayor parte de los estudiantes obtuvo los puntajes más altos,
permitiendoles alcanzar los niveles 6 y 7 de calificación en este
componente de evaluación. De igual manera, los trabajos internos de
\textbf{Español}, las notas obtenidas por los estudiantes, aunque
presentan una alta dispersión, en general fueron satisfactorias.

Por otro lado, en el caso de \textbf{Frances \emph{ab initio}}, aunque
la nota final de la asignatura fue satisfactoria, la dispersión de los
datos en el trabajo interno muestra que hay una disparidad en la
capacidad y alcance de los estudiantes en la realización de este trabajo
(evaluación oral individual).

Los trabajos internos de la asignatura de \textbf{Historia} estuvieron
muy por debajo de los resultados esperados y por lo tanto, de los
niveles de calificación más altos;, lo que afectó mientras que en el
caso de \textbf{Política Global}, la mayoría de estudiantes obtuvo
resultados por encima de la media en la escala de calificación, lo que
les permitió estar en los niveles 4 y 5 de su trabajo interno.

En el caso de las ciencias \textbf{Física} tuvo los mejores resultados
en el desarrollo de su trabajo de investigación, si bien la dispersión
de las calificaciones es amplia y requiere una mejora en el proceso de
calificación y estandarización. Los resultados promedio de los trabajos
de \textbf{Computer Science} presentaron una amplia disparidad, y
estuvieron en promedio 15 puntos por debajo del máximo de 34, lo que
influyó a que no se obtuvieran mejores notas en la clase.

Finalmente, \textbf{Matemáticas} y \textbf{Química} presentaron los
resultados más bajos en sus trabajos internos. En el primer caso, las
investigaciones presentadas estuvieron casi 10 puntos por debajo del
máximo de 24, lo que terminó por hacer que los estudiantes obtuvieran
calificaciones en los niveles 2 y 3 (de un máximo de 7) para este
componente.

Aunque hubo estudiantes que obtuvieron muy buenas calificaciones en el
trabajo interno de matemáticas, la dispersión de los puntos obtenidos
indica una amplia variabilidad en la comprensión, la investigación y la
ejecución de este trabajo.

\begin{tcolorbox}[enhanced jigsaw, colbacktitle=quarto-callout-warning-color!10!white, toprule=.15mm, colback=white, toptitle=1mm, leftrule=.75mm, rightrule=.15mm, left=2mm, arc=.35mm, colframe=quarto-callout-warning-color-frame, breakable, bottomtitle=1mm, titlerule=0mm, opacityback=0, title=\textcolor{quarto-callout-warning-color}{\faExclamationTriangle}\hspace{0.5em}{Advertencia}, coltitle=black, opacitybacktitle=0.6, bottomrule=.15mm]
Es fundamental que se revisen los procesos de planeación y práctica
pedagógica que se desarrollen para la realización de los
\textbf{trabajos internos}. Es necesario que se haga énfasis en la
comprensión de sus objetivos y criterios de evaluación. Dentro de la
planeación debe plantearse no solo la presentación y el trabajo
individual en la realización de los trabajos, sino que debe haber una
\textbf{modelación estricta} y un \textbf{acompañamiento eficaz} en cada
asignatura para la realización de este componente.

Los departamentos deben hacer visibles estratégias pedagógicas para una
calificación más precisa de cada uno de los criterios establecidos para
los trabajos internos en cada asignatura. Por ejemplo, utilizar los
criterios de evaluación para diseñar instrumentos (desempeños) que
permitan a los estudiantes comprender los requerimientos en cada uno de
ellos, y a los profesores, ejercitar su capacidad de calificación.
\end{tcolorbox}



\end{document}
